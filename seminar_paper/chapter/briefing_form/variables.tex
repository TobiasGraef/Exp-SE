% !TEX root = ../../seminar.tex

\subsubsection{Variables}

Variables are operationalizations of concepts, and there are usually three kinds of variables: Dependent, independent and controlled variables \cite{BuddiesVariables,Seltman2015}. A good variable must be measurable by any means \cite{BuddiesVariables}. Seltman additionally defines several qualities that make a good variable: \obsrvQuote{high reliability, absence of bias, low cost, practicability, objectivity, high acceptance, and high concept validity}\cite[p. 10]{Seltman2015}. \\

\todosoft{include classification of statistical type. has influence on choice of statistical method in experiment.}

We decided to include the different variables in the briefing form to make the decomposition of hypotheses easier while searching for related work. In order to make the search more fertile. Additionally the variables make it easier to evaluate the validity of the evidence.



\todo{maybe put this whole description part of experiment already in workflow and just state here that it is vital for the briefing.}
