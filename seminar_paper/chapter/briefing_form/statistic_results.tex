% !TEX root = ../../seminar.tex

\subsubsection{Statistical Results}
\label{statisticalresults}

The outcome of an experiment is important for researchers who search for related work. Additionally the results help validating the solidity of evidence found. The statistic result of an experiment should never contain any interpretation. Interpretations should be given in a conclusion or discussion section separately from the data.

This field should at least contain the used statistic measurement, method, or model and their resulting parameters (e.g. Analysis of Variance: $F(2.57)=211.496, p<.001)$. It is advisable to use established methods to ensure comparability and reproducibility. More on statistical analysis can be found in books like \cite{Wohlin2012}, or \cite{Albert2008} as already mentioned in Section \ref{subsec:designing conducting and interpreting experiment}.
