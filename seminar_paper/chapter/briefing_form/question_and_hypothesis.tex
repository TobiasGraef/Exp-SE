% !TEX root = ../../seminar.tex

\subsection{Research Question and Hypothesis}
\label{subsec:research question and hypothesis}

In many cases  question and hypothesis  are rather redundant. Nonetheless, both are included in \briefingform. The research question is included, because researchers use to search for existing evidence by entering their question into search engines. 
Whereas the hypothesis has its right to exist in \briefingform to support evaluation of relevance and validity of found evidence. 

Therefore the form should be filled in with the research question and hypothesis constructed earlier (see section \ref{subsec:formulating research question and hypothesis}). The research question contains at least technology, context and effect. The hypothesis is in form of a testable prediction (e.g., \q{If [I do X], then [Y] will happen.} \cite{Buddies2010}).