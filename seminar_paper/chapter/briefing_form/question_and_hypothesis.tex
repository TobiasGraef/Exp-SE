\subsection{Research Question and Hypothesis}
\label{subsec:research question and hypothesis}

We decided to include both research question and hypothesis in the briefing form. Despite them being rather redundant in many cases. The research question is included due to the method researchers use to search for existing evidence, by entering their question into search engines. On the other side the hypothesis has its right to exist in the briefing form to support evaluation of validity and relevance of found evidence. 

Therefore the fields in the form should be filled in with the research question and hypothesis constructed earlier (see section \ref{subsec:formulating research question and hypothesis}). The research question in a asking form which contains technology, context and effect. Or it is constructed according to the PICOT criteria (see fig. \ref{fig:PICOT}). And the hypothesis in form of a testable prediction (e.g., \q{If [I do X], then [Y] will happen.} \cite{Buddies2010}).