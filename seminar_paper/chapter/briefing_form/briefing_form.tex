% !TEX root = ../../seminar.tex

\todo{remove repetition in heading if we do not find better name for briefing form.}
\section{Briefing Form - \briefingform}
\label{sec:briefing form}

In this section, the \textit{\briefingform} is introduced. The \briefingform is a one page sheet designed for two purposes: Guide users through the design of an experiment and make conducted experiments easier skimmable and searchable. It is similar to SEED but has a more detailed structure to support the search for existing evidence even more.\\
Experiments are used to obtain scientific evidence. To make the evidence as reliable as possible, the experiment needs to be designed and conducted with minimal flaws. That can be a very difficult task because experiments and their interpretation can be tremendously prone to errors or mistakes. Especially for people new to experimenting, an awareness for common mistakes and best practices can be very beneficial. By supporting the researcher to understand the study thoroughly, mistakes can be discovered early in the process.\\
On a small scale, the Briefing Form is also meant to be a  supporting framework for a systematic workflow.\\
For software practitioners, it is important to find solutions for a problem quickly. Reading through papers can be very time consuming. The Briefing Form can help speed up the search by providing a clear structure for a quick overview for an experiment.\\
Since the \briefingform~is meant to implicitly guide the user's approach to experimenting and prevent mistakes, we tried to address problems observed  by Rainer \etal \cite{Rainer2006}. The issues and their respective conclusions are listed in Table \ref{table:issuesEBSE}. The \briefingform consists of three parts: Research Question/Hypothesis, Experiment and Conclusion. These are explained in the following.\\

\todosoft{ supports researcher in understanding the scope of research?}
%\todo{list: contains of: research question, hypothesis, experiment/context or deduction, conclusion
%experiment contains: independent and dependent variables (control variables), method, results?}