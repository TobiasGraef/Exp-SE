% !TEX root = ../../seminar.tex

\section{Briefing Form - \briefingform}
\label{sec:briefing form}

In this section, the \textit{Briefing Form} is introduced. The Briefing Form is a sheet designed for two purposes: Guide users through the design of an experiment and make conducted experiments easier skimmable and searchable.\\
Experiments are used to obtain scientific evidence. To make the evidence as reliable as possible, the experiment needs to be designed and conducted with minimal flaws. That can be a very difficult task because experiments and their interpretation can be tremendously prone to errors or mistakes. Especially for people new to experimenting, an awareness for common mistakes and best practices can be very benefitial.
The Briefing Form is meant to be a  supporting framework for a systematic workflow. It is supposed to shift the focus to the important aspects in each step and sensitize for critical errors.\\
For software practitioners, it is important to find solutions for a problem quickly. Reading through papers can be very time consuming. The Briefing Form can help speed up the search by providing a clear structure for a quick overview for an experiment.\\
Since the \briefingform is meant to implicitly guide the user's approach to experimenting and prevent mistakes, we tried to address problems observed  by Rainer et al. \cite{Rainer2006}. The issues and their respective conclusions are listed in Table \ref{table:issuesEBSE}.