% !TEX root = ../../seminar.tex

\todo{remove repetition in heading if we do not find better name for briefing form.}
\section{Briefing Form - \briefingform}
\label{sec:briefing form}

In this section, \textit{\briefingform} is introduced. It is a one page sheet designed for two purposes: Guiding users through the design of an experiment and provide summaries of experiments that are easier searchable and screenable. It is similar to SEED but has a more detailed structure to additionally support the search for existing evidence.

Experiments are used to obtain scientific evidence. To increase the reliability of evidence, experiments needs to be designed and conducted with minimal flaws. This can be a very difficult task because experiments and their interpretation can be prone to errors or mistakes. Especially for people new to experimenting, an awareness for common mistakes and best practices can be very beneficial. By supporting the researcher in understanding the study thoroughly mistakes can be discovered early in the process.\\
On a small scale, \briefingform is also meant to be a  supporting framework for a systematic workflow.

For software practitioners, it is important to quickly find solutions for a problem. Reading through papers can be very time consuming. \briefingform can help speed up the search by providing a clear structure for a quick overview of an experiment.

Since \briefingform is meant to implicitly guide the user through experimenting and thereby preventing mistakes. It consists of three parts: Research Question/Hypothesis, Experiment and Conclusion. These are explained in the following.