\begin{itemize}
\item importance: Abstracts, together with the title, are used to identify relevant research, not only in SLRs. Often the abstract is the only part of the paper that can be accessed for free. Therefore abstract and title should contain all necessary information to decide whether a paper (in case of SLR: primary study) is relevant in this context. (TODO source: \q{Procedures for Undertaking Systematic Reviews})
	$\rightarrow$ quality of abstract crucial for research, how to support researcher in writing useful abstracts? Structured Abstracts provide guidance for writer and reader. 
\item Suggestion of elements proposed by Jedlitschka et al.\cite{Jedlitschka2008}: 
	\begin{enumerate}
		\item Background or Context: motivation for conducting the study, previous research
		\item Objective or Aim: Object that is studied, focus and perspective of the study, hypothesis
		\item Method: e.g. experimental design, participants and selection criteria, measurement and analyzing technique...
		\item Results: most important findings (treatment outcome), no interpretation!
		\item Limitations: scope of study, limits of generalization (often as part of conclusion)
		\item Conclusion: Interpretation of results, put results in context
	\end{enumerate}
	(short and early version:\cite{Jedlitschka2005})
	\newline
	Elements similar to \q{IMRAD}-structure (see paper \q{Adoption of structured abstracts by general medical journals and format for a structured abstract})
\item About completeness and clarity of structured abstracts:
	\begin{enumerate}
	\item Structured abstracts include more relevant information and are easier to read than conventional abstracts. \cite{Budgen2008} \cite{Budgen2007}
	\item Inexperienced authors are likely to produce clearer and more complete abstracts when using a structured form.\cite{Budgen2011} 
	\item On average structured abstracts are longer and have better readability than unstructured abstracts. \cite{KBO2008}
	\item these findings are in accordance with the ones in other disciplines (\red{which?, source})
	\end{enumerate}
\item guidelines for construcing structured abstract (from unstructered ones) in \cite{KBO2008}
\item Guide with examples (psychology): \q{how to write a good abstract for scientigic paper.}, C. Andrade
	\newline
	ANSI/NISO Z39.14.1997 (R2015) Guidelines for Abstracts
\item use standard terminology (commonly used industry terms) \cite{Jedlitschka2008}
\item structured abstracts are longer and often size is limited (journals): prioritize traditional elements, still structured: background (one sentence), objective, method, results, and conclusion \cite{Jedlitschka2008}
\end{itemize}