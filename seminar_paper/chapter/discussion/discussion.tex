% !TEX root = ../../seminar.tex

\section{Discussion}
\label{sec:discussion}

To support students in scientific working, this paper introduced a process along with guiding documents.
The process is mainly based on EBSE.\\
The workflow and documents were not evaluated in this work. Therefore, they need to be tested in future work. This should be done with students in real world scenarios such as bachelor and master theses. Comparing multiple theses is a difficult task. There are many influencing factors and already a certain variance in the quality of theses requiring a lot of subjects. Also finding a suiting metric to classify and compare two theses that might not be particularly similar is not trivial. Whereas asking students in a qualitative study can provide useful insight whether the documents were actually helpful or not.\\
When formatively evaluated, a digitalized version of \briefingform should be implemented to allow state of the art data collection and searching. This is a step towards an infrastructure which supports the use of EBSE on a larger scale. That kind of infrastructure might also be of interest for software practitioners and scientists.