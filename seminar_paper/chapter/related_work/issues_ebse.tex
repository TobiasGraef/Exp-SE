% !TEX root = ../../seminar.tex
\\
\todo{make table environment}:
\todosoft{bolder appearance, better quatability}
\begin{enumerate}
	\item \label{itm:issue1} \q{Students had problems constructing well-formulated EBSE questions.}
	\item \label{itm:issue2} \q{Students used limited criteria for identifying the best or better evidence [\ldots]}
	\item \label{itm:issue3} \q{Students used a very limited number of search terms.}
	\item \label{itm:issue4} \q{Students provided poor explanation in their reports of how their searches were conducted.}
	\item \label{itm:issue5} \q{Students varied in their use of the EBSE checklist.}
	\item \label{itm:issue6} \q{Some students critically appraised the technologies rather than the publications (evidence) on the technologies.}
	\item \label{itm:issue7} \q{But we also think that the kinds of problems students were tackling [\ldots] are not the kinds of problems researchers commonly investigate.}
\end{enumerate}

% \ref{table:issuesEBSE}.
\todo{Break up table}
%\begin{center}
	\begin{table}
	\begin{tabular}{ | p{6cm} | p{7cm} |}
	\hline
	\textbf{Observation} & \textbf{Conclusion/Guideline} \\ \hline
	
	\obsrvQuote{Students had problems constructing well-formulated EBSE questions.} (p. 6) 
	& Give examples for good questions to make sure the user understands a good question's scope of 
	information. Also, explicitly list which building blocks should be contained in the question. \\ \hline
	
	\obsrvQuote{Students used limited criteria for identifying the best or better evidence[...]} (p. 6) 
	& Support decision-making to get a decision as unbiased and suited as possible.
	Since a decision's quality is highly dependent on the individual case, we only give a very general hint to
	the user. The idea is to sensitize the user to consciously prevent bias as good as possible. \\ \hline
	
	\obsrvQuote{Students used a very limited number of search terms.} (p. 6) 
	& If users look for something very specific without knowing the technical term, search engines might yield
	better results when used with more detailed search terms.
	Also, synonyms or similar words might widen the search's scope to find more related work.
	Encourage more search terms by providing examples containing enough search terms. \\ \hline
	
	\obsrvQuote{Students provided poor explanation in their reports of how their searches were conducted.}
	(p. 7)
	& Recommend users to write down their search terms for a more structured search approach.  \\ \hline
	
	\obsrvQuote{Students varied in their use of the EBSE checklist.} (p. 7)
	& Design the checklist in a way to support the user's workflow instead of hindering it.
	Keep it as simple as possible and provide enough examples to make the user never guess an item's 
	meaning. \\ \hline
	
	\obsrvQuote{Some students critically appraise the technologies rather than the publications (evidence) on
	the technologies} (p. 7)
	& \todo{link ciritcal appraisal checklist} \\ \hline
	
	\obsrvQuote{But we also think that the kinds of problems students were tackling [...] are not the kinds of 
	problems researchers commonly investigate.} (p. 8)
	& Scientific and practical evidence can have very different requirements regarding content and other
	aspects such as duration of evaluation. Due to the large differences, we kept the forms and checklists 
	rather general to make the useful for both sides. \\ \hline
	\end{tabular}
	\caption{\todo{Issues with EBSE found by Rainer \etal\cite{Rainer2006}}}
	\label{table:issuesEBSE}
	\end{table}
%\end{center}