% !TEX root = ../../seminar.tex

\section{Related Work}
\label{sec:related work}
\todosoft{einleitung}

Rainer \etal \cite{Rainer2006} released a paper about the reported use of EBSE by 15 under-graduate students. 
We have used observations listed in the paper to create design guidelines for the documents introduced in section \ref{sec:research process} and \ref{sec:briefing form} . There are seven main issues we tried to address:
% !TEX root = ../../seminar.tex
\begin{table}
	\begin{enumerate}
		\item \label{itm:issue1} \q{Students had problems constructing well-formulated EBSE questions.}
		\item \label{itm:issue2} \q{Students used limited criteria for identifying the best or better evidence [\ldots]}
		\item \label{itm:issue3} \q{Students used a very limited number of search terms.}
		\item \label{itm:issue4} \q{Students provided poor explanation in their reports of how their searches were conducted.}
		\item \label{itm:issue5} \q{Students varied in their use of the EBSE checklist.}
		\item \label{itm:issue6} \q{Some students critically appraised the technologies rather than the publications (evidence) on the technologies.}
		\item \label{itm:issue7} \q{But we also think that the kinds of problems students were tackling [\ldots] are not the kinds of problems researchers commonly investigate.}
	\end{enumerate}
	\caption{Issues with EBSE found by Rainer \etal\cite{Rainer2006}}
	\label{table:issuesEBSE}
\end{table}

\todo{mention? some of these issues addressed in \q{Using systematic reviews and evidence-based software engineering with masters students}}\\

SEED \cite{Janzen2008} and the \emph{Evidence Map} \cite{EBSEWeb} are databases that are meant to ease the access to studies.\\ 
SEED is a community-driven online database that has been created during graduate courses. It lists summaries of studies in a common format and grouped by topics. The studies are added by researchers, whereas users create summaries, ratings and comparison grids. There only exists a prototype \cite{SEED} that has not been updated since 2009.\todo{criticise format of study representation?}\\
The Evidence Map is primarily concerned with  classifying systematic literature reviews (secondary studies, see sections \ref{subsec:EBSE} and \ref{subsec:search for existing evidence}), but also provides lists of primary and tertiary studies as well as information and guidelines about EBSE. A major disadvantage is the lack of study summaries. It has not been updated since 2012.

\todo{Large graph} \cite{Rainer2008}
our graph: tried to offer better orientation within the graph/prcoesss. to allow the students to get familiar with the underlying process and not get stuck with details that require more experinece to be solved within a reasonable time frame.