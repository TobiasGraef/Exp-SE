\subsubsection{Hypothesis}

For each quantitative research question there should be a hypothesis - an educated guess about the outcome of the research question \cite{Buddies2010,Farrugia2009}. A good hypothesis needs to be a testable, prediction of the studies outcome, but it is important that it does not contain any interpretation \cite{Prasad2001}. A simple template for writing a hypothesis would be:
\begin{quote}
	If [I do this], then [this] will happen. \cite{Buddies2010}
\end{quote}
Vickers \emph{et al.} propose a more refined structure, whereas a good hypothesis needs to include three components: Two or more variables, population/context, and the relationship between the variables \cite{Vickers}. {\color{red} TODO specify the thing with the variables more.} For example a good hypothesis in software engineering research could be: 
\begin{quote}
	{\color{red}Pair programming used by professional software developers improves code quality, in comparison to teams that use conventional techniques. (revise this)}
\end{quote}
Furthermore, when conducting empirical research {\color{red}- as we propose in this paper - (revise this)} the hypothesis should be formulated as a \emph{null hypothesis $H_0$}, and be accompanied by an \emph{alternative hypothesis $H_1$} \cite{Farrugia2009}. The null hypothesis is a theory that is believed to be true but not proven jet. The alternative hypothesis is the opposite prediction of the null hypothesis \cite{Prasad2001}. At the end of the study the null hypothesis is empirically tested, and only if it is rejected (i.e., there is a significant difference between groups) the alternative hypothesis is taken as true. This confirms that effects did not show by chance alone \cite{Farrugia2009}. A null hypothesis to the example above would be:
\begin{quote}
	{\color{red}Pair programming used by professional software developers does not affect code quality. (revise this)}
\end{quote}
To support the validity of the study even more, the hypotheses should be formulated as 2-sided hypothesis. \obsrvQuote{A 2-sided hypothesis states that there is a difference between [groups, but without specifying the direction of the outcome].}\cite[p.280]{Farrugia2009} 1-sided hypotheses should only be used when there is a strong justification for one direction of the outcome \cite{Farrugia2009}. A 2-sided revision of the $H_1$ from above would be: 
\begin{quote}
	{\color{red}Pair programming used by professional software developers does affect code quality. (revise tis)}	
\end{quote}
{\color{red}(TODO specify more tips for writing a good hypothesis. Creswell2014)}  