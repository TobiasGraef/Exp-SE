% !TEX root = ../../seminar.tex

\subsection{Answer Question}
\label{subsec:answer question}

Before you can answer a question the existing studies must be critically appraised. This means checking the study design, study quality, relevance for the research question, as well as consistency between different studies. 

Rainer \etal found that \q{students use limited criteria for identifying the best or better evidence} \cite{Rainer2006} (see issue \ref{itm:issue2} in section \ref{sec:related work})\todo{how to refer to the issues?}. They suggest sensitizing students for biases to solve this issue. We also suggest guiding the user through the process, because students had additional problems (see issue \ref{itm:issue6}). \todosoft{rephrase}

The \emph{GRADE approach} \cite{Atkins2004} is a grading system for studies that goes beyond simple hierarchies of study types. It was introduced for medicine and has been used in the software engineering domain \cite{Wohlin2013EvidenceProfile,Dyba2008}. It is a well-defined method and differentiates between the quality of evidence and the strength of recommendation. Figure \ref{fig:critical appraisal} shows a checklist containing the important factors to consider when appraising a study and we suggest following it when appraising studies. \todo{ref issue \ref{itm:issue6} in table \ref{table:issuesEBSE}}

\begin{figure}[h]
\fbox{\parbox{\textwidth}{
\large
\textbf{Study Appraisal Checklist}
\normalsize
\begin{enumerate}
\item \textbf{Is there any vested interest?}
	\begin{itemize}
	\item Who sponsored the study?
	\item Do the researchers have any vested interest in the results?
	\end{itemize}
\item \textbf{Is the evidence valid?}
	\begin{itemize}
	\item Was the study’s design appropriate to answer the question?
	\item How were the tasks, subjects, and setting selected?
	\item What data was collected, and what were the methods for collecting the data?
	\item Which methods of data analysis were used, and were they appropriate?
	\end{itemize}
\item \textbf{Is the evidence important?}
	\begin{itemize}
	\item What were the study’s results?
	\item Are the results credible, and, if so, how accurate are they?
	\item What conclusions were drawn, and are they justified by the results?
	\item Are the results of practical and statistical significance?
	\end{itemize}
\item \textbf{Can the evidence be used in practice?}
	\begin{itemize}
	\item Are the study’s findings transferable to other industrial settings?
	\item Did the study evaluate all the important outcome measures?
	\item Does the study provide guidelines for practice based on the results?
	\item Are the guidelines well described and easy to use?
	\item Will the benefits of using the guidelines outweigh the costs?
	\end{itemize}
\item \textbf{Is the evidence in this study consistent with the evidence in other available studies?}
	\begin{itemize}
	\item Are there good reasons for any apparent inconsistencies?
	\item Have the reasons for any disagreements been investigated?
	\end{itemize}
\end{enumerate}
}}
\caption{Checklist for critical appraisal of studies compiled by Dyb{\aa} \etal \cite{Dyba2005} \todo{image}}
\label{fig:critical appraisal}
\end{figure}

One important aspect of appraising studies is checking whether they might be biased. J{\o}rgensen \etal \cite{Jorgensen2016} reported indications of research and publication bias being quite common in the domain of software engineering. Shepperd makes similar findings and gives a good and short overview of the problem \cite{Shepperd2015}.

The hypothesis is accepted or rejected on the basis of the critical appraisal and the research question is answered accordingly.


\todosoft{- maybe reference to different types of bias}
