% !TEX root = ../../seminar.tex

\subsection{Search for Existing Evidence}
\label{subsec:search for existing evidence} 

After formulating a research question it is important to know which scientific results exist that help answering this question. Students normally perform an \q{ad-hoc} literature review that is prone to missing evidence and getting biased results (experimenter bias).

For that reason EBSE proposes the use of \emph{systematic literature reviews} (SLR), a type of secondary study, to search for studies related to the research question. The aim of SLRs is to \q{identify, assess and combine the evidence from primary research studies using an explicit and rigorous method} \cite{Zhang2011}. Using this well-defined method might prevent biased results of the literature review, but requires more effort - both in time and skill - than traditional literature reviews. There are guidelines \cite{keele2007,Wohlin2014,Zhang2011} and reports of experiences with conducting SLRs \cite{Brereton2007} that provide the means necessary to conduct a SLR. In any case published SLRs are very useful. They provide a summary of multiple studies concerning one research question even if conducting one is not feasible. Currently there are at least two projects concerned with making SLRs and finding relevant studies easier by collecting and indexing them. Namely SEED \cite{Janzen2008} and Evidence Map \cite{EBSEWeb}. Both have not been updated recently. \todosoft{more disadv?}

Students voted the SLR conducted during the EBSE process as one of the hardest steps \cite{keele2007}. Therefore, in this work a less formal search process to save time and effort is suggested. But it should be noted, that the search should still be planned and structured to get the desired results \todosoft{state it?}. If it is possible conducting a SLR is always preferable.
\newline
\newline
Nevertheless, several guidelines from SLRs still can be used:
\begin{itemize}
\item Think about search strings and search engines beforehand and write them down. This also addresses issue \ref{itm:issue4} in table \ref{table:issuesEBSE}.

\item It is unlikely to find all relevant literature using only a single search engine \cite{Brereton2007}.

\item \emph{Snowballing} is useful, both in a forward and backward manner \cite{Wohlin2014}. Keywords found during this process, can also be used to refine the search. This addresses issue \ref{itm:issue3} in table \ref{table:issuesEBSE}. Usually, search engines provide these features.\\
\emph{Backward}-snowballing refers to looking at the references of a publication and therefore going back in time.\\
\emph{Forward}-snowballing means going forward in time by looking at the papers citing the current paper.

\item The search strategy depends on the research question. If the research domain provides a vast amount of studies the search can be restricted, for example by excluding older studies \cite{Brereton2007}.

\item Even for more experienced students it might be necessary to refine the research question based on the search result. Because their understanding of the research domain increases \cite{Brereton2007}. For example having only a few search result can be caused by research questions that are too narrow. Research Questions that are too general can lead to an abundance of search result which would take too long to examine.

\item Categorize your research question using classification systems like the \emph{ACM Computing Classification System}\todo{ref to url} and search within these categories. This is especially helpful if it is problematic to limit the scope in the research area or there are uncertainties about wording and naming.
\end{itemize} 

\todosoft{
mention? systematic mapping studies (broader, but not as deep as SLRs (deep: quantitative analysis and quality assessment))
\q{Systematic Mapping Studies in
Software Engineering}, Petersen \etal 2008
}



