\subsection{Search for Existing Evidence}
\label{subsec:search for existing evidence} 

After formulating a research question it is important to know which scientific results exist that help answer this question.

EBSE proposes the use of systematic literature reviews (SLR)to search for studies related to the research question. The aim of SLRs is to \q{identify, assess and combine the evidence from primary research studies using an explicit and rigorous method}\todo{quote \q{Identifying relevant studies in software engineering}, Zhang et al. 2010}. Using this well-defined methods might prevent biased results of the literature review, but requires more effort than traditional literature reviews.
In any case published SLRs are very useful, even if conducting one is not feasible, as they provide a summary of multiple studies concerning one research question. There exist guidelines \todo{cite: Guidelines for performing systematic literature reviews in software engineering, Guidelines for
Snowballing in Systematic Literature Studies and a Replication in Software Engineering, Identifying relevant studies in software engineering, EBSE guidelines?} and reports of experiences with conducting SLRs \todo{cite: Lessons from applying the systematic literature review process within the software engineering domain}.

\todo{fill the gap?}
The search strategy depends on the research question. If the research domain provides a vast amount of studies the search can \todo{should, must?} be restricted, for example the exclusion of older studies. \todo{cite \q{Lessons from applying the systematic literature review process within the software engineering domain}}
\todo{fill the gap?}
Even for more experienced students it might be necessary to refine the research question based on the search result, because their understanding of the research domain increases. For example having only a few search result can be caused by research questions that are too narrow. Research Questions that are too general can lead to an abundance of search result that would take too long to examine \todo{or can't be handled?}. 
\newline
\newline
in form of study: SLR. use existing ones, or even conduct yourself? (time consuming, no easy task)
\newline
$\rightarrow$ database like SEED useful
\newline
\q{Identifying relevant studies in software engineering}, Zhang et al. 2010
\begin{itemize} 
	\item hints:
	\item use google scholar ("wrapper" for IEEExplore, ACM ...?) $\rightarrow$forward / backward search
	\item use categories (e.g. CCS), especially if not sure about wording / boundaries of topic, 
\end{itemize}