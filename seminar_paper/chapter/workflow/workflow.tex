% !TEX root = ../../seminar.tex

\section{Student Process Improvement Document - \checklist}
\label{sec:research process}

\begin{minipage}{\linewidth}
\begin{wrapfigure}{R}{0.3\textwidth}
	\centering
	\vspace{-1.0cm}
	\includegraphics[trim={3cm 0 3cm 0}, height=14.55cm]{figures/workflow_graph.pdf}
	\caption{Workflow Graph}
	\label{fig:workflow_graph}
\end{wrapfigure}


In this section, a document called \emph{Student Process Improvement Document (\checklist{})} is introduced. It is supposed to guide students through scientific working with EBSE in mind.

Rainer \etal found, that \q{[s]tudents varied in their use of the EBSE checklist} \cite{Rainer2006} (see issue \ref{itm:issue5} in table \ref{table:issuesEBSE}). Therefore, an important design criteria for \checklist{} is ease of use through clear and simple instructions. Especially tailored for students with little knowledge about scientific working in general.

The process of \checklist{} contains eight steps:
\begin{enumerate}
\item Formulate question
\item Formulate hypothesis
\item Search for existing evidence
\item Substantial evidence
\item Experiment
\item Answer question
\item Discussion
\item Evaluate process
\end{enumerate}


The whole graph can be seen in figure \ref{fig:workflow_graph}. The actual document can be found in appendix \ref{appendix:checklist}.\\
On the left of the document, a flow chart of the proposed process is depicted. For computer science students this should be a fast way to navigate through and orient themselves in the process. To further assist navigation visually, each process step has been assigned a unique color. This color schemes reoccurs in the tools section of the document as well as in \briefingform.

On the right additional information is given. Each process step in \checklist{} contains a short description, some guidelines, and acceptance criteria. The guidelines contain methods and tools on how to process the current step. The acceptance criteria give students orientation on when a step is completed.
\end{minipage}









