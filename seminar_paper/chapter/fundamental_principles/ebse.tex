% !TEX root = ../../seminar.tex

\subsection{EBSE - Evidence-Based Software Engineering}
\label{subsec:EBSE}

The aim of EBSE is \q{to improve decision making related to software development and maintenance by integrating current best evidence from research with practical experience and human values} \cite{Dyba2005}. Practising EBSE includes five steps:
\begin{enumerate}
	\item Ask an answerable question.
	\item Find the best evidence that answers that question.
	\item Critically appraise this evidence.
	\item Apply the evidence (and critical appraisal).
	\item Evaluate the performance in previous steps.
\end{enumerate}
Formulating the question precisely is important for the success of the process. The question should be formulated broad enough, so important studies are not missed, but must be precise enough to cope with the amount of studies (see section \ref{subsec:formulating research question and hypothesis}).

In medicine, researchers rely heavily on already published \emph{systematic literature reviews} (SLR) to find relevant studies. SLRs try to identify and interpret all available literature regarding a specific research question \cite{keele2007}. There are several organisations dedicated to conduct such reviews in medicine. The lack of this infrastructure makes applying the evidence-based approach in software engineering more difficult, but the number of existing SLRs increases steadily. For further reading on SLR, see Wohlin \etal \cite{Wohlin2012}.

It is important to check the quality of the identified studies, because being published is not a guarantee for absence of errors. Sometimes the integrity of results is compromised, because the research method has weaknesses or the researcher has vested interest connected to the outcome of the study.

Applying the evaluated evidence means integrating it with personal experience and other requirements. This step highly depends on the context and type of technology under evaluation.

At last the performance in previous steps is evaluated to improve applications  of the EBSE process in the future.

Kitchenham \etal also identify two major problems inherent to software engineering:
\begin{enumerate}
\item The skill factor: Performing software engineering methods and techniques often require skilled practitioners. This prevents blinding and can therefore cause problems related to subject and experimenter bias.
\item The lifecycle issue: Prediction of behaviour of deployed technology is difficult and it is hard to isolate effects because of interaction with other methods and technologies.
\end{enumerate}

Furthermore, they also state two approaches to reduce each of these effects \cite{EBSE}.

