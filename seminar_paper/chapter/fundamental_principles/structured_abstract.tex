% !TEX root = ../../seminar.tex

\subsection{Structured Abstract}
\label{subsec:structured abstract}

In their guidelines for reporting experiments in software engineering Jedlitschka \etal \cite{Jedlitschka2008} propose the use of \emph{structured abstracts}. They adopted the idea from medicine and psychology, where structured abstracts were introduced to increase the quality of abstracts. Structured abstracts guide the writer as well as the reader by using headings. Although a variety of different elements is used (see for example Nakayama \etal \cite{nakayama2005adoption}), the most common elements of structured abstracts are \emph{Background/Context}, \emph{Objective/Aim}, \emph{Methods}, \emph{Results} and \emph{Conclusion/Discussion}. Jedlitschka \etal \cite{Jedlitschka2008} suggest the use of a sixth heading called \emph{limitations}. This information is necessary to decide whether a result can be transfered to another context. Kitchenham \etal \cite{KBO2008} include this information in the \emph{conclusion} section.

The list and description of elements below closely follows the suggestion of Jedlitschka \etal \cite{Jedlitschka2008}:
\begin{enumerate}
	\item \emph{Background/Context}: Briefly explains the motivation for conducting the study and refers to previous research.
	\item \emph{Objective/Aim}: Describes the purpose of the study, including the object that is studied as well as focus and perspective. This part should cover the research question.
	\item \emph{Methods}: Sums up used research methods. For example experimental design, setting, participants and selection criteria, intervention and measurement and analyzing technique.
	\item \emph{Results}: The key findings are described here in form of numerical values. Do not include interpretations here. See section \ref{statisticalresults} (Statistical Results) for further details.
	\item \emph{Limitations}: Describes the scope  of the study to point out the limits of generalization. This element might be incorporated in the \emph{conclusion}-element.
	\item \emph{Conclusion}:  Contains the interpretation of results and puts them into larger context.
\end{enumerate}

%Additionally, keywords can be used for a quick overview. They can also help to find work easier using search engines.

There are several studies comparing structured and unstructured abstracts in the domain of software engineering with regard to completeness and clarity:
\begin{itemize}
\item Structured abstracts include more relevant information and are easier to read than conventional abstracts \cite{Budgen2007,Budgen2008}.
\item Inexperienced authors are likely to produce clearer and more complete abstracts when using a structured form \cite{Budgen2011} .
\item On average structured abstracts are longer (limitations in conclusion is a good idea to prevent lengthy abstracts) and have better readability than unstructured abstracts \cite{KBO2008}.
\end{itemize}
	
These findings are consistent with the ones in other disciplines. It is important to mention that there are critics of structured abstracts that are supported by studies, but in general structured abstracts are considered advantageous \cite{hartley2004,hartley2014}.	

A downside to structured abstracts is their length compared to unstructured ones. If the size of abstracts is limited, the abstract should still be in a structured form traditional elements should be prioritized: background (one sentence), objective, methods, result and conclusion \cite{Jedlitschka2008}.

If it is not possible to structure an abstract (e.g. due to standard of journal or supervisor, length limitations) the elements of a structured abstracts should be contained in the unstructured abstract to make sure no important information is missing. See the common structure of the clearest abstracts as found by Shaw \cite{shaw2003}. Moreover the reader should be able to quickly identify each element by reading through the abstract.

For further information  and examples of structured abstracts see the guide of C. Andrade \cite{Andrade2011} and Kitchenham \etal \cite{EBSE}.
