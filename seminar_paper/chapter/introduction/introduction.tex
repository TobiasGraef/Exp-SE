% !TEX root = ../../seminar.tex


\section{Introduction}
\label{sec:introduction}

Our advisor observed that most software build in a bachelor's or master's thesis is poorly evaluated or not evaluated at all. Controlled studies are needed to attain valid measurements that allow comparison. Students often struggle with obtaining this \emph{empirical evidence} (see table \ref{table:issuesEBSE}) leading to unusable results making a proper comparison difficult or not possible. Furthermore, large pieces of software are hard to evaluate because effects can be hard to isolate.\\
Therefore, the aim of this work is to create a supporting system for students helping to improve the quality (in particular the substantiality) of their studies in the domain of software engineering.

The final system intended is a database containing a digitalized collection of experiments. The system is meant to simplify searching, scanning and comparing experiments. Allowing to quickly find existing evidence that can be used for software design decisions. To populate that collection, a process to create evidence correctly and a compact representation of the results are needed. In this work, two documents are proposed: \emph{\checklist} to guide students through the scientific process and \emph{\briefingform} for a compact and structure resume of experiments.

First we align the proposed process with related work in section \ref{sec:related work}. In section \ref{sec:fundamental principles} \emph{evidence-based software engineering} (EBSE) and structured abstracts are introduced. EBSE is the foundation of the process introduced later in section \ref{sec:research process}. Therefore, the idea behind EBSE is very fundamental  for this paper.
Structured abstracts are used as a base for the form introduced in section \ref{sec:briefing form}.













