% !TEX root = ../../seminar.tex


\section{Introduction}

motivation

\begin{itemize}
\item observation: most software build in a bachelor or master thesis is poorly evaluated (if it is done at all)
\item Initial aim: Find/create a method to evaluate and compare software developed by students.
\item But: 1. Large pieces of software are hard to evaluate because effects can be hard to isolate.
\item 2.  To find out wich software is more suitable for a given problem, measurements are needed for a quantitative comparison that can be treated as empirical evidence.
\item To attain valid measurements that allow comparison, a controlled study is needed.
Students sometimes struggle with making that kind of controlled study \todo{tabellenverweis} leading to unusable results making a proper comparison difficult or not possible at all.
\item So we revised the question to:
How to create a supporting system for students helping to improve the quality (in particular the substantiality) of their studies in the domain of software engineering.

% \item > Evidence based software engineering 
% \item aim: aid students in conducting a similar approach. Tools: guidelines, checklists
\end{itemize}

The final system we intend is an electronical database containing a collection of experiments. The system is meant to simplify searching, scanning and comparison of experiments. Allowing to quickly find existing evidence that can be used for software design decisions. To populate that collection, a process to create evidence correctly and a compact representation of the results are needed. We propose two documents: \emph{\checklist} to guide students through the scientific process and \emph{\briefingform} for a compact, structured and complete resume of the conducted experiment. 



\begin{itemize}
\item experimentation/empiricism in CS?
\item \q{should computer scientists experiment more?, Tichy 1997}
\item Experimental evaluation in CS: A quantitative study, 1994: 40\% of papers requiring quant evaluation have none (other disciplins: 15\%)
\item \q{A survey of Controlled Experiments in SE, 2005} from 1993 to 2002: 1,9\% of software engineering paper report controlled experiments
\end{itemize}


















