\subsection{Fundamental Principles}

short introduction to EBSE,..

\begin{itemize}
	\item adopted from medicine (EBM), starting point: decisions in SE often not based on evidence for suitability, quality.., thus increasing the risk of poor decisions \red{in motivation?}
	\item \textbf{Evidence-based approach}: integrate all available research (evidence) in decision making process
	\item \textbf{Aim}: \q{EBSE aims to improve decision making related to software development and maintenance by integrating current best evidence from research with practical experience and human values.} \cite{Dyba2005}
	\item \textbf{Five steps} of practising EBSE \cite{EBSE}:
		\begin{enumerate}
			\item Ask an answerable question.
			\item Find the best evidence that answers that question.
			\item Critically appraise this evidence.
			\item Apply the evidence (and critical appraisal).
			\item Evaluate the performance in previous steps.
		\end{enumerate}
		$\rightarrow$ important tool: Systematic Literature Review (SLR)
	\item \textbf{SLR} \cite{keele2007}: identify and interpret all available literature regarding a research question
		$\rightarrow$ papers should be written for synthesis \red{(TODO requirements for this, common mistakes/problems?)}
	\item Problems inherent to SE\cite{EBSE}:
		\begin{enumerate}
			\item Skill factor: performing SE methods and techniques often require skilled practitioners. This prevents blinding and can therefor cause problems related to subect and experimenter bias. (2 approaches to reduce these effects in \cite{EBSE})
			\item lifecycle issue: prediction of behaviour (long time?) of deployed technology difficult, hard to isolate efffects because of interaction with other methods/technologies (also 2 approaches to cope with these effects in \cite{EBSE})
		\end{enumerate}
		\item Step 2: SLR in SE: lack of systematic reviews \red{(still correct? source?)}, lack of replication studies \red{(source?)}, problems regarding SLR \red{TODO}
\end{itemize}