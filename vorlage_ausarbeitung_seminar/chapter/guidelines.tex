% Set root file for TeXShop:
% !TEX root = ../seminar.tex

\newcommand{\obsrvQuote}[1]{\textit{``#1''} }


\section{Our Approach/Guidelines}

\subsection{Setting up our design guidelines}

The checklist (TODO name?) is meant to implicitly guide the user's approach to experimenting.
By guiding the user, typical mistakes might be prevented.
To create guidelines that help preventing typical users' mistakes, these mistakes first need to be identified.
In this section, experiences and guidelines found in related work are discussed.
The conclusions are used as basis for design of our guidelines.
The first set of guidelines is based on the report of Rainer et al. \cite{Rainer2006}:

\begin{center}
	\begin{tabular}{ | p{6cm} | p{7cm} |}
	\hline
	\textbf{Observation} & \textbf{Conclusion/Guideline} \\ \hline
	
	\obsrvQuote{Students had problems constructing well-formulated EBSE questions.} (p. 6) 
	& Give examples for good questions to make sure the user understands a good question's scope of 
	information. Also, explicitly list which building blocks should be contained in the question. \\ \hline
	
	\obsrvQuote{Students used limited criteria for identifying the best or better evidence[...]} (p. 6) 
	& Support decision-making to get a decision as unbiased and suited as possible.
	Since a decision's quality is highly dependent on the individual case, we only give a very general hint to
	the user. The idea is to sensitize the user to consciously prevent bias as good as possible. \\ \hline
	
	\obsrvQuote{Students used a very limited number of search terms.} (p. 6) 
	& If users look for something very specific without knowing the technical term, search engines might yield
	better results when used with more detailed search terms.
	Also, synonyms or similar words might widen the search's scope to find more related work.
	Encourage more search terms by providing examples containing enough search terms. \\ \hline
	
	\obsrvQuote{Students provided poor explanation in their reports of how their searches were conducted.}
	(p. 7)
	& TODO \\ \hline
	
	\obsrvQuote{Students varied in their use of the EBSE checklist.} (p. 7)
	& Design the checklist in a way to support the user's workflow instead of hindering it.
	Keep it possibly simple and provide enough examples to make the user never guess an item's 
	meaning. \\ \hline
	
	\obsrvQuote{Some students critically appraise the technologies rather than the publications (evidence) on
	the technologies} (p. 7)
	& TODO Give a hint/indication? \\ \hline
	
	\obsrvQuote{But we also think that the kinds of problems students were tackling [...] are not the kinds of 
	problems researchers commonly investigate.} (p. 8)
	& Scientific and practical evidence can have very different requirements regarding content and other
	aspects such as duration of evaluation. To limit this paper's scope, we focus on scientific evidence. \\ \hline
	\end{tabular}
\end{center}


\newpage
\subsection{Research Question, Hypothesis, and Objectives}

For researchers to produce relevant results and understand their research domain fully, the step of developing a good research question, with a supporting hypothesis and sometimes objectives is integral \cite{Farrugia2009}. These three components should be carefully designed before conducting the study that tries to answer the question. Otherwise it is more likely to produce questions that are already answered, or \obsrvQuote{could potentially lead to spuriously positive findings of association through chance alone.} \cite[p. 280]{Farrugia2009} 

\subsubsection{Research Question}

\begin{itemize}
	\item What ist it?
	\item For what is it?
	\item Qualitative/Quantitaive -> Quantitative mostly in SE so we concentrate on quantitative. {\color{red} (find source)}
	\item Types of research questions {\color{red} (Shaw)}
	\item How to research question:
		\begin{itemize}
			\item FINER \& PICOT  {\color{red} (Farrugia et al.)}
			\item Intervention/Contect/Effect  {\color{red} (Kitchenham et al. 2005)}
			\item Research Design Book  {\color{red} (Creswell)}
		\end{itemize}
\end{itemize}




\subsubsection{Hypothesis}

\begin{itemize}
	\item What is it?
	\item For what is it?
	\item educated guess {\color{red} (Science Buddies, Prasad)}
	\item is testable statement  {\color{red} (Prasad)}
	\item is a prediction {\color{red} (Prasad)}
	\item no interpretation {\color{red}(Prasad)}
	\item relationship between two or more variables {\color{red} (Vickers)}
	\item Only in Quantitative research  {\color{red} (Creswell)}
	\item two-sided/one-sided hypothesis -> use only two-sided  {\color{red} (Farrugia et al.)}
	\item Null hypothesis in empirical work  {\color{red} (Farrugia et al.)}
	\item contains varaibales/population/relationship  {\color{red} (Vickers)}
	\item How to hypothesis:
	\begin{itemize}
		\item  {\color{red} (Prasad)}
		\item from websites  {\color{red} (Science Buddies)}
		\item  {\color{red} (Creswell)}
	\end{itemize}
\end{itemize}
\subsubsection{Objectives}

 Sometimes researchers define objectives to their hypotheses. They are active statements that \obsrvQuote{define specific aims of the study and should be clearly stated} \cite[p. 280]{Farrugia2009} at the beginning of research. Objectives help to define the study (e.g. helping to calculate sample size). \cite{Farrugia2009,Vickers} Although we do not include objectives in our \briefingform we would like to mention them for reasons of completeness.

\newpage
\subsection{Workflow}
% Picture's style
\tikzstyle{decision} = [diamond, draw, fill=gray!20, text width=4.5em, text badly centered, inner sep=0pt]
\tikzstyle{evidence} = [ellipse, draw, fill=red!20,  minimum width=25em, minimum height=4em, text badly centered, inner sep=0pt]
\tikzstyle{block} = [rectangle, draw, fill=gray!20, text width=5em, text centered, rounded corners, minimum height=4em]
\tikzstyle{line} = [draw, -latex']


\begin{tikzpicture}[node distance = 3cm, auto]
	\node [block] (hypothesis) {Create Hypothesis};
	\node [block, right of = hypothesis] (research) {Search for existing evidence};
	\node [decision, right of = research] (enough) {Enough evidence?};
	\node [block, right of = enough] (experiment) {Experiment};
	\node [evidence, below of = enough] (evidence) {Evidence Pool};
	\node [block, right of = experiment] (evaluate) {Evaluate};


	\path [line] (hypothesis) -- (research);
	\path [line] (research) -- (enough);
	\path [line] (enough) -- node {no}  (experiment) ;
	\path [line] (enough.north)  -| node [near start] {yes} (evaluate.north);
	\path [line] (experiment) edge [bend left] (enough);

	
	\path [line, dashed, color=red] (research) edge [bend right] node {grows} (evidence);
	\path [line, dashed, color=red] (experiment) -- node {grows}  (evidence);
	\path [line, dashed, color=red] (evidence) edge [bend right] (evaluate);
\end{tikzpicture}

TODO:\\
Add numbers to each node, explain each node\\
Missing node: "Make Decision" at the end?\\
Layout/Style/Color


\newcommand{\exampleQuote}[1]{\small{\textit{``#1''}}}

\subsection{Checklist}

{\large\textbf{Question}}\\
Contains \textbf{\textit{technology}} in a \textbf{\textit{context}} showing an \textbf{\textit{effect}}. TODO Kitchenham Quote (practitioners)?\\
\exampleQuote{Does pair programming in professional software development teams increases code quality?}
\\
\\
\\
{\large\textbf{Hypothesis}}\\
Needs to contain a \textbf{\textit{prediction}} and needs to be \textbf{\textit{testable}}.\\
\exampleQuote{If you do x, then y will happen}
\\
\\
\\
{\large\textbf{Experiment}}\\
\textbf{Context}\\
\\
\textbf{Dependent Variables}\\
Variables that are  \textbf{\textit{measured}} during the study.\\
\\
\textbf{Independent Variables}\\
Variables that are  \textbf{\textit{changed}} during the study.\\
\\
\textbf{Method}\\
Lab-/Field study, number of participants, metrics, ...\\
\\
\textbf{Results}\\
Experiment's outcome\\
\textbf{No} interpretation or conclusion!\\
\\
\\
\\
{\large\textbf{Conclusion}}\\
Interpretation of experiment's results.\\
Verifying or Falsifying Hypothesis.\\
Scope of generalization.








