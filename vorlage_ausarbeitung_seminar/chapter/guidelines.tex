% Set root file for TeXShop:
% !TEX root = ../seminar.tex

\newcommand{\obsrvQuote}[1]{\textit{``#1''} }

\section{Our Approach/Guidelines}

%============================%
\subsection{Setting up our design guidelines}

The checklist (TODO name?) is meant to implicitly guide the user's approach to experimenting.
By guiding the user, typical mistakes might be prevented.
To create guidelines that help preventing typical users' mistakes, these mistakes first need to be identified.
In this section, experiences and guidelines found in related work are discussed.
The conclusions are used as basis for design of our guidelines.
The first set of guidelines is based on the report of Rainer et al. \cite{Rainer2006}:

\begin{center}
	\begin{tabular}{ | p{6cm} | p{7cm} |}
	\hline
	\textbf{Observation} & \textbf{Conclusion/Guideline} \\ \hline
	
	\obsrvQuote{Students had problems constructing well-formulated EBSE questions.} (p. 6) 
	& Give examples for good questions to make sure the user understands a good question's scope of 
	information. Also, explicitly list which building blocks should be contained in the question. \\ \hline
	
	\obsrvQuote{Students used limited criteria for identifying the best or better evidence[...]} (p. 6) 
	& Support decision-making to get a decision as unbiased and suited as possible.
	Since a decision's quality is highly dependent on the individual case, we only give a very general hint to
	the user. The idea is to sensitize the user to consciously prevent bias as good as possible. \\ \hline
	
	\obsrvQuote{Students used a very limited number of search terms.} (p. 6) 
	& If users look for something very specific without knowing the technical term, search engines might yield
	better results when used with more detailed search terms.
	Also, synonyms or similar words might widen the search's scope to find more related work.
	Encourage more search terms by providing examples containing enough search terms. \\ \hline
	
	\obsrvQuote{Students provided poor explanation in their reports of how their searches were conducted.}
	(p. 7)
	& TODO \\ \hline
	
	\obsrvQuote{Students varied in their use of the EBSE checklist.} (p. 7)
	& Design the checklist in a way to support the user's workflow instead of hindering it.
	Keep it possibly simple and provide enough examples to make the user never guess an item's 
	meaning. \\ \hline
	
	\obsrvQuote{Some students critically appraise the technologies rather than the publications (evidence) on
	the technologies} (p. 7)
	& TODO Give a hint/indication? \\ \hline
	
	\obsrvQuote{But we also think that the kinds of problems students were tackling [...] are not the kinds of 
	problems researchers commonly investigate.} (p. 8)
	& Scientific and practical evidence can have very different requirements regarding content and other
	aspects such as duration of evaluation. To limit this paper's scope, we focus on scientific evidence. \\ \hline
	\end{tabular}
\end{center}


\newpage
%============================%
\subsection{Research Question, Hypothesis, and Objectives}

For researchers to produce relevant results and understand their research domain fully, the step of developing a good research question, with a supporting hypothesis and sometimes objectives is integral \cite{Farrugia2009}. These three components should be carefully designed \emph{before} conducting the study that tries to answer the question. Otherwise it is more likely to produce questions that are already answered, or \obsrvQuote{could potentially lead to spuriously positive findings of association through chance alone.} \cite[p. 280]{Farrugia2009} 

\subsubsection{Research Question}

The question the later study is designed to answer is called research question \cite{Vickers}. It should be an answerable question and address a relevant issue in the research area \cite{Dyba2005}. Preceding a research question is the need for a deep understanding of the topics that have already been studied, in order to produce questions which drive knowledge further. The questions that arise during the acquisition of knowledge, and cannot be answered by means of EBSE, are likely appropriate questions for further research \cite{Farrugia2009}. \newline
There are two general classes of research questions: qualitative and quantitative questions. Qualitative research states questions which report, describe, or explore a subject \cite[p. 139-141]{Creswell2014}. In computer science as the research field matures these questions become more and more rare {\color{red} (find source)}. Therefore focus is on quantitative research questions in this paper. \obsrvQuote{Quantitative research questions inquire about the relationships among variables} \cite[p. 143]{Creswell2014}, and from them emerge quantitative hypotheses. \newline
To understand the structure of research questions Shaw provides a model where she categorizes research questions from software engineering papers in five types \cite{Shaw2002} {\color{red} (maybe cut out)}. \newline
To design a good research question Haynes coined the acronym PICO: Population, Intervention, Comparison group, and Outcome \cite{BrianHaynes2006}. Sometimes Time is added as fifth component, when it is important over what time frame the study is conducted, see Table \ref{table:PICOT}. A research question structured with the PICOT approach supports in restricting the research question and steers thereby hypotheses and study. By restricting the research question researchers can limit bias and increase the internal validity of the study, but a too narrow question may also lead to decreased external validity \cite{Farrugia2009}. \\
Before PICOT Sackett and colleagues suggested that good research questions consist out of three components: Intervention, Context and Outcome \cite{Sackett2000}, which is a more coarse grained decomposition than PICOT. Dyb{\aa} \emph{et al.} displayed a fitting example for this template in software engineering: \obsrvQuote{Does pair programming lead to improved code quality when practiced by professional software developers?} \cite[p. 60]{Dyba2005} Here the intervention is pair programming, the context of interest are professional software developers, and the outcome is improved code quality \cite{Dyba2005}. To verify the quality of a freshly designed research question Hulley \emph{et al.} suggest the use of the FINER criteria. It highlights key aspects of the question and provides thereby new angles to view the proposed study from. The FINER criteria consists of: Feasible, Interesting, Novel, Ethical, and Relevant \cite{Farrugia2009}. A more detailed view of the FINER criteria can be seen in Table \ref{table:FINER}. {\color{red}(TODO specify more tips for writing a good question. Creswell2014)}  


\begin{table}[]
	\centering
	\fbox{
		\begin{tabular}{ p{3em} p{6.5em} p{22em} }
			\textbf{P}	& Population & What specific population are you interested in? \vspace{1em}\\
			
			\textbf{I}	& Intervention (technology) & What is the investigational technology/ intervention? \vspace{1em}\\
			
			\textbf{C}	& Comparison group & What is the main alternative/baseline to compare with the intervention \vspace{1em}\\ 
			
			\textbf{O}	& Outcome & What do you intend to accomplish, measure, improve or affect? \vspace{1em}\\ 
			
			\textbf{T}	& Time & What is the appropriate follow-up time to assess outcome? 
		\end{tabular}	
	}
	\vspace{1em}
	\caption{PICOT criteria adjusted to fit better in computer science research.}
	\label{table:PICOT}
\end{table}

\begin{table}[]
	\centering
	\fbox{
		\begin{tabular}{ p{3em} p{6em} p{22em} }
				\textbf{F}	& Feasible & \tabitem Adequate number of subjects \\
				\multicolumn{2}{l}{} & \tabitem Adequate technical expertise \\
				\multicolumn{2}{l}{} & \tabitem Affordable in time and money \\
				\multicolumn{2}{l}{} & \tabitem Manageable in scope \vspace{1em}\\
				\textbf{I}	& Interesting & \tabitem Getting the answer intrigues investigator, peers and community \vspace{1em}\\
	
				\textbf{N}	& Novel & \tabitem Confirms, refutes or extends previous findings \vspace{1em}\\
	
				\textbf{E}	& Ethical & \tabitem Amendable to a study that institutional review board will approve \vspace{1em}\\ 
				\textbf{R}	& Relevant & \tabitem To scientific knowledge \\
				\multicolumn{2}{l}{} & \tabitem To clinical and health policy \\
				\multicolumn{2}{l}{} & \tabitem To future research
		\end{tabular}	
	}
	\vspace{1em}
	\caption{FINER criteria for a good research question \cite{Farrugia2009}}
	\label{table:FINER}
\end{table}




\subsubsection{Hypothesis}

\begin{itemize}
	\item What is it?
	\item For what is it?
	\item is testable statement  {\color{red} (find source)}
	\item Only in Quantitative research  {\color{red} (Creswell)}
	\item two-sided/one-sided hypothesis -> use only two-sided  {\color{red} (Farrugia et al.)}
	\item Null hypothesis in empirical work  {\color{red} (Farrugia et al.)}
	\item contains varaibales/population/relationship  {\color{red} (health website)}
	\item How to hypothesis:
	\begin{itemize}
		\item  {\color{red} (Prasad)}
		\item from websites  {\color{red} (find sources)}
		\item  {\color{red} (Creswell)}
	\end{itemize}
\end{itemize}
\subsubsection{Objectives}

 Sometimes researchers define objectives to their hypotheses. They are active statements that \obsrvQuote{define specific aims of the study and should be clearly stated} \cite[p. 280]{Farrugia2009} at the beginning of research. Objectives help to define the study (e.g. helping to calculate sample size). \cite{Farrugia2009,Vickers} Although we do not include objectives in our \briefingform we would like to mention them for reasons of completeness.

\newpage
%============================%
\subsection{Workflow}
% Picture's style
\tikzstyle{decision} = [diamond, draw, fill=gray!20, text width=4.5em, text badly centered, inner sep=0pt]
\tikzstyle{evidence} = [ellipse, draw, fill=red!20,  minimum width=25em, minimum height=4em, text badly centered, inner sep=0pt]
\tikzstyle{block} = [rectangle, draw, fill=gray!20, text width=5em, text centered, rounded corners, minimum height=4em]
\tikzstyle{line} = [draw, -latex']


\begin{tikzpicture}[node distance = 3cm, auto]
	\node [block] (hypothesis) {Create Hypothesis};
	\node [block, right of = hypothesis] (research) {Search for existing evidence};
	\node [decision, right of = research] (enough) {Enough evidence?};
	\node [block, right of = enough] (experiment) {Experiment};
	\node [evidence, below of = enough] (evidence) {Evidence Pool};
	\node [block, right of = experiment] (evaluate) {Evaluate};


	\path [line] (hypothesis) -- (research);
	\path [line] (research) -- (enough);
	\path [line] (enough) -- node {no}  (experiment) ;
	\path [line] (enough.north)  -| node [near start] {yes} (evaluate.north);
	\path [line] (experiment) edge [bend left] (enough);

	
	\path [line, dashed, color=red] (research) edge [bend right] node {grows} (evidence);
	\path [line, dashed, color=red] (experiment) -- node {grows}  (evidence);
	\path [line, dashed, color=red] (evidence) edge [bend right] (evaluate);
\end{tikzpicture}

TODO:\\
Add numbers to each node, explain each node\\
Missing node: "Make Decision" at the end?\\
Layout/Style/Color


\newcommand{\exampleQuote}[1]{\small{\textit{``#1''}}}

\subsection{Checklist}

{\large\textbf{Question}}\\
Contains \textbf{\textit{technology}} in a \textbf{\textit{context}} showing an \textbf{\textit{effect}}. TODO Kitchenham Quote (practitioners)?\\
\exampleQuote{Does pair programming in professional software development teams increases code quality?}
\\
\\
\\
{\large\textbf{Hypothesis}}\\
Needs to contain a \textbf{\textit{prediction}} and needs to be \textbf{\textit{testable}}.\\
\exampleQuote{If you do x, then y will happen}
\\
\\
\\
{\large\textbf{Experiment}}\\
\textbf{Context}\\
\\
\textbf{Dependent Variables}\\
Variables that are  \textbf{\textit{measured}} during the study.\\
\\
\textbf{Independent Variables}\\
Variables that are  \textbf{\textit{changed}} during the study.\\
\\
\textbf{Method}\\
Lab-/Field study, number of participants, metrics, ...\\
\\
\textbf{Results}\\
Experiment's outcome\\
\textbf{No} interpretation or conclusion!\\
\\
\\
\\
{\large\textbf{Conclusion}}\\
Interpretation of experiment's results.\\
Verifying or Falsifying Hypothesis.\\
Scope of generalization.

\newpage
%============================%
\subsection{Structured abstracts in software engineering}


\begin{itemize}
\item importance: Abstracts, together with the title, are used to identify relevant research, not only in SLRs. Often the abstract is the only part of the paper that can be accessed for free. Therefore abstract and title should contain all necessary information to decide whether a paper (in case of SLR: primary study) is relevant in this context. (TODO source: \q{Procedures for Undertaking Systematic Reviews})
	$\rightarrow$ quality of abstract crucial for research, how to support researcher in writing useful abstracts? Structured Abstracts provide guidance for writer and reader. 
\item Suggestion of elements proposed by Jedlitschka et al.\cite{Jedlitschka2008}: 
	\begin{enumerate}
		\item Background or Context: motivation for conducting the study, previous research
		\item Objective or Aim: Object that is studied, focus and perspective of the study, hypothesis
		\item Method: e.g. experimental design, participants and selection criteria, measurement and analyzing technique...
		\item Results: most important findings (treatment outcome), no interpretation!
		\item Limitations: scope of study, limits of generalization (often as part of conclusion)
		\item Conclusion: Interpretation of results, put results in context
	\end{enumerate}
	(short and early version:\cite{Jedlitschka2005})
\item About completeness and clarity of structured abstracts:
	\begin{enumerate}
	\item Structured abstracts include more relevant information and are easier to read than conventional abstracts. \cite{Budgen2008} \cite{Budgen2007}
	\item Inexperienced authors are likely to produce clearer and more complete abstracts when using a structured form.\cite{Budgen2011} 
	\item On average structured abstracts are longer and have better readability than unstructured abstracts. \cite{KBO2008}
	\end{enumerate}
\item guidelines for construcing structured abstract (from unstructered ones) in \cite{KBO2008}
\end{itemize}





